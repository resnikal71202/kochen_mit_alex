\section{Zwiebelkuchen}
\begin{wraptable}{r}{0.4\textwidth}
    \centering
    \begin{tabularx}{0.39\textwidth}{|l|X|}
        \hline
        Menge & Zutaten: \\
        \hline
        2x & Pizzateig \\
        \hline
        3 Becher & Crème fraîche mit Kreutern \\
        \hline
        2x & Gratinkäse Käse\\
        \hline
    \end{tabularx}
\end{wraptable}
Um einen köstlichen Zwiebelkuchen zuzubereiten, beginnen Sie damit, den Backofen auf 200 °C vorzuheizen (Ober-/Unterhitze). 
Anschließend schälen Sie die Zwiebeln und schneiden sie in kleine Quadrate. 
Rollen Sie den Pizzateig aus und legen Sie ihn auf ein Backblech. 
Bestreichen Sie den Teig großzügig mit Crème fraîche, die mit Kräutern gemischt ist. 
Verteilen Sie die Zwiebelstücke gleichmäßig auf der Crème fraîche. 
Als nächstes streuen Sie großzügig Gratinkäse über die Zwiebeln. 
Backen Sie den Zwiebelkuchen im vorgeheizten Backofen für etwa 25–30 Minuten, bis er goldbraun und knusprig ist. 
Nehmen Sie den Zwiebelkuchen aus dem Ofen und lassen Sie ihn etwas abkühlen, bevor Sie ihn servieren. 
Dieser klassische Zwiebelkuchen ist perfekt als Vorspeise, Snack oder Hauptgericht.
\newpage
