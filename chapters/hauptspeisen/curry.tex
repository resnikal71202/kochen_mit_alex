\documentclass[../../book.tex]{subfiles}
\begin{document}
\section{Curry}
\begin{wraptable}{r}{0.4\textwidth}
  \centering
  \begin{tabularx}{0.39\textwidth}{|l|X|}
    \toprule
    Menge & Zutaten: \\
    \midrule
    200g & Sojagranulat \\
    \midrule
    400g & Tofu \\
    \midrule
    2x & Kokosmilch \\
    \midrule
    400g & S\"uskartoffeln \\
    \midrule
    400g & Brokkoli \\
    \midrule
    1x & Zwiebel \\
    \midrule
    1x & Paprika \\
    \midrule
    1x & Zucchini \\
    \midrule
    & Kurkumapulver \\
    \midrule
    & Currypaste (mild) \\
    \midrule
    & Spinat \\
    \midrule
    & Reis \\
    \midrule
    & Erbsen \\
    \midrule
    & Öl \\
    \bottomrule
  \end{tabularx}
\end{wraptable}
Zuerst das Sojagranulat in warmem Wasser einweichen und den Tofu in Würfel schneiden. Die Süßkartoffeln schälen und in kleine Stücke schneiden. Den Brokkoli in Röschen teilen. Die Zwiebel, Paprika und Zucchini klein schneiden.

Erhitzen Sie etwas Öl in einer großen Pfanne. Braten Sie die Zwiebel darin an, bis sie glasig ist. Fügen Sie dann die Currypaste und das Kurkumapulver hinzu und rösten Sie es kurz an, um die Aromen freizusetzen.

Geben Sie das eingeweichte Sojagranulat hinzu und vermengen Sie alles gut. Fügen Sie dann den Tofu, Süßkartoffeln, Brokkoli, Paprika und Zucchini hinzu. Braten Sie alles an, bis das Gemüse etwas weich wird.

Gießen Sie die Kokosmilch in die Pfanne und lassen Sie das Curry köcheln, bis das Gemüse gar ist und die Soße eine cremige Konsistenz hat. Rühren Sie gelegentlich um, damit nichts anbrennt.

Fügen Sie den Spinat, die Erbsen und nach Geschmack weitere Gewürze hinzu. Rühren Sie gut um und lassen Sie das Curry kurz köcheln, bis der Spinat zusammenfällt und die Erbsen durchgewärmt sind.

Servieren Sie das Curry heiß zusammen mit Reis als Beilage.
\newpage
\end{document}
