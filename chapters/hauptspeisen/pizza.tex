\documentclass[../../book.tex]{subfiles}
\begin{document}
\section{Pizza}
\begin{wraptable}{r}{0.4\textwidth}
    \centering
    \begin{tabularx}{0.39\textwidth}{|l|X|}
        \hline
        Menge & Zutaten: \\
        \hline
        1x & Pizzateig \\
        \hline
        1x & Pastierte Tomaten \\
        \hline
        1x & Gratinkäse Käse\\
        \hline
        1x & Paprika\\
        \hline
        1x & Zwiebel\\
        \hline
        1x Schale & Pilze\\
        \hline
        & Oliven\\
        \hline
        

    \end{tabularx}
\end{wraptable}
Zunächst den Backofen auf 200 °C vorheizen. Anschließend den Pizzateig ausrollen und auf ein Backblech legen. Dann die Tomatensauce auf dem Teig verteilen und dabei darauf achten, dass der Rand nicht zu dick belegt wird.

Das Gemüse nach Belieben schneiden und auf der Tomatensauce verteilen. Sie können hierbei gerne Ihrer Kreativität freien Lauf lassen und verschiedene Gemüsesorten miteinander kombinieren.

Als nächstes den geriebenen Käse auf dem Gemüse verteilen. Sie können auch hier nach Belieben variieren und z.B. Mozzarella oder einen anderen Käse verwenden.

Optional können Sie nun noch Gewürze wie Oregano, Basilikum und Knoblauchpulver über die Pizza streuen, um ihr noch mehr Geschmack zu verleihen.

Die Pizza wird nun im vorgeheizten Ofen für ca. 10-15 Minuten gebacken, bis der Käse geschmolzen und der Teig knusprig ist. Achten Sie darauf, dass die Pizza nicht verbrennt.

Wenn die Pizza fertig gebacken ist, nehmen Sie sie aus dem Ofen, schneiden sie in Stücke und servieren sie heiß und frisch. Ihre vegetarische Pizza ist nun bereit zum Genießen!

Guten Appetit!
\newpage
\end{document}