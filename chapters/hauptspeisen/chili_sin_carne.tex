\documentclass[../../book.tex]{subfiles}
\begin{document}
\section{Chili sin carne}
\begin{wraptable}{r}{0.4\textwidth}
  \centering
  \begin{tabularx}{0.39\textwidth}{|l|X|}
    \toprule
    Menge & Zutaten: \\
    \midrule
    200g? & Sojagranulat \\
    \midrule
    & Gemüsebrühe\\
    \midrule
    1x & Passierte Tomaten \\
    \midrule
    1x Dose & Mais\\
    \midrule
    1x Dose & Kidneybohnen\\
    \midrule
    4x & Kartoffeln\\
    \midrule
    2x & Zwiebeln\\
    \midrule
    1-2 & Chilischoten\\
    \midrule
    & Brot\\
    \midrule
    & Gewürze nach Wahl:\\
    & Currypulver\\
    & Paprikapulver\\
    & Chilipulver\\
    \bottomrule
  \end{tabularx}
\end{wraptable}

Zuerst schälen wir die Kartoffeln und kochen sie.

Währenddessen bringen wir Wasser zum Kochen und fügen Gemüsebrühe-Pulver hinzu. Mit dieser Brühe wird das Sojagranulat eingeweicht. Nach 10-15 Minuten Einweichzeit wird das Granulat durch ein Sieb abgetropft.

In der Zwischenzeit schälen wir die Zwiebeln und schwitzen sie in einer Pfanne an. Anschließend geben wir das abgetropfte Sojagranulat hinzu und braten es mit an.

In einem großen Topf mischen wir die passierten Tomaten, Mais und Kidneybohnen und geben dann die angebratenen Zwiebeln und das Sojagranulat hinzu. Dann schneiden wir die gekochten Kartoffeln in kleine Stücke und geben sie ebenfalls in den Topf.

Das Ganze lassen wir nun auf mittlerer Hitze köcheln und würzen es nach Belieben mit den Gewürzen. Die Chilischoten können ebenfalls klein geschnitten und hinzugefügt werden, um dem Gericht eine gewisse Schärfe zu verleihen.

Zum Schluss servieren wir das fertige Gericht mit Brot und genießen es am besten in Gesellschaft von Freunden und Familie.

\newpage
\end{document}