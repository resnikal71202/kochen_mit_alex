\documentclass[../../book.tex]{subfiles}
\begin{document}
\section{Pasta mit Aubergine und Kirschtomaten}

\begin{figure*}[h]
  \centering
  \includegraphics[width=\textwidth]{media/pasta_aubergine_tomate.jpg}
\end{figure*}

\begin{wraptable}{r}{0.4\textwidth}
  \centering
  \begin{tabularx}{0.39\textwidth}{|l|X|}
    \toprule
    Menge & Zutaten: \\
    \midrule
    500 g & Spaghetti \\
    \midrule
    1 & Aubergine \\
    \midrule
    500 g & Kirschtomaten \\
    \midrule
    100 g & Kalamata-Oliven \\
    \midrule
    4 EL & Kapern \\
    \midrule
    1 & Knoblauchzehe \\
    \midrule
    2 EL & Olivenöl \\
    \midrule
    Nach Geschmack & Salz, Pfeffer \\
    \midrule
    Nach Bedarf & Frischer Basilikum \\
    \bottomrule
  \end{tabularx}
\end{wraptable}

Für eine mediterrane Pasta mit Aubergine und Kirschtomaten beginnen Sie damit, die Spaghetti nach Packungsanweisung in Salzwasser al dente zu kochen. Währenddessen die Aubergine in Würfel schneiden, die Kirschtomaten halbieren, den Knoblauch fein hacken und die Kalamata-Oliven sowie die Kapern abtropfen lassen.

Erhitzen Sie das Olivenöl in einer großen Pfanne und braten Sie die Auberginenwürfel mit etwas Salz etwa 5 Minuten scharf an. Fügen Sie den gehackten Knoblauch hinzu und braten Sie ihn weitere 2 Minuten mit. Geben Sie anschließend die Oliven und Kapern hinzu und schwenken Sie sie kurz mit, bevor Sie die Kirschtomaten hinzufügen.

Sobald die Spaghetti fertig sind, gießen Sie sie ab und vermengen Sie sie in der Pfanne mit dem Gemüse. Schmecken Sie das Gericht mit Salz und Pfeffer ab und garnieren Sie es nach Belieben mit frischem Basilikum. Dieses einfache und doch geschmackvolle Gericht bringt ein Stück Italien auf Ihren Teller.
\newpage
\end{document}