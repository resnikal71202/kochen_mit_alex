\documentclass[../../book.tex]{subfiles}
\begin{document}
\section{Bananen-Schoko-Muffins}

\begin{wraptable}{r}{0.4\textwidth}
  \centering
  \begin{tabularx}{0.39\textwidth}{|l|X|}
    \toprule
    Menge & Zutaten: \\
    \midrule
    4 & reife Bananen (braune bis schwarze Flecken ideal) \\
    \midrule
    2 & Eier (Größe M) \\
    \midrule
    190 g & Mehl (Type 405 oder 550) \\
    \midrule
    130 g & Zucker \\
    \midrule
    100 g & Schokotröpfchen (oder gehackte Schokolade) \\
    \midrule
    80 ml & Pflanzenöl (neutral, z. B. Sonnenblumenöl) \\
    \midrule
    1,5 TL & Backpulver (ca. 6 g) \\
    \bottomrule
  \end{tabularx}
\end{wraptable}

Um die Bananen-Schoko-Muffins zuzubereiten, benötigst Du ein 12er-Muffinblech und 12 Papierbackförmchen.

Heize zunächst den Backofen auf 180 °C (Ober-/Unterhitze) vor.

In einer Schüssel vermengst Du 190 g Mehl, 1,5 TL (ca. 6 g) Backpulver und 100 g Schokotröpfchen. Zerdrücke die Bananen mit einer Gabel auf einem Teller.

In einer separaten Schüssel verquirlst Du die 2 Eier mit einem Handrührgerät oder einer Küchenmaschine. Rühre anschließend 130 g Zucker und 80 ml Pflanzenöl ca. 1 Minute unter.

Füge die zerdrückten Bananen hinzu und mische kurz. Hebe danach die Mehlmischung vorsichtig unter, bis sie gerade vermengt ist. Achtung: Nicht zu lange rühren, damit der Teig fluffig bleibt.

Lege das Muffinblech mit den Papierförmchen aus. Verteile den Teig gleichmäßig in die Förmchen, am besten mit einem Eisportionierer oder zwei Esslöffeln. Befülle sie bis ca. 0,5 cm unter den Rand.

Backe die Muffins auf der mittleren Schiene für 20–25 Minuten. Je nach Reifegrad der Bananen kann der Teig flüssiger sein, was die Muffins saftiger, aber auch etwas kompakter macht.

Lass die Muffins etwas abkühlen und genieße sie frisch!

\end{document}