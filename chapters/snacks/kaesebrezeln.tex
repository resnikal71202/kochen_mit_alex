\documentclass[../../book.tex]{subfiles}
\begin{document}
\section{Käsebrezeln}
\begin{wraptable}{r}{0.4\textwidth}
  \centering
  \begin{tabularx}{0.39\textwidth}{|l|X|}
    \toprule
    Menge & Zutaten: \\
    \midrule
    4x & Brezeln (frisch oder tiefgekühlt) \\
    \midrule
    150g & Käse (z.B. Gouda oder Emmentaler) \\
    \midrule
    & Butter \\
    \midrule
    & Salz \\
    \midrule
    & Pfeffer \\
    \bottomrule
  \end{tabularx}
\end{wraptable}

Falls tiefgekühlte Brezeln verwendet werden, diese nach Packungsanweisung auftauen und vorbacken.

Den Backofen auf 180°C (Ober-/Unterhitze) vorheizen. Ein Backblech mit Backpapier auslegen.

Die Brezeln auf das Backblech legen. Falls gewünscht, mit etwas Butter bestreichen. Den Käse grob reiben und gleichmäßig auf den Brezeln verteilen.

Die Käsebrezeln für ca. 10–15 Minuten backen, bis der Käse geschmolzen und goldbraun ist.

Nach Belieben mit Salz und Pfeffer würzen und heiß servieren.

\newpage
\end{document}